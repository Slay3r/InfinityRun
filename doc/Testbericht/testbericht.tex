% Vorlage fuer Handouts
% zum Seminar "Diskrete Geometrie und Kombinatorik -- ein topologischer Zugang"
% im WS 2008/09
%
%%%%%%%%%%%%%%%%%%%%%%%%%%%%%%%%%%%%%%%%%%%%%%%%%%%%%%%%%%%%%%%%%%%%%%%%%%%%
% Allgemeine Hinweise
% - Halten Sie den LaTeX-Code so uebersichtlich wie moeglich;
%   (La)TeX-Fehlermeldungen sind oft kryptisch -- in einem ordentlich 
%   strukturierten Quellcode lassen sich Fehler leichter finden und 
%   beseitigen
%
%
%%%%%%%%%%%%%%%%%%%%%%%%%%%%%%%%%%%%%%%%%%%%%%%%%%%%%%%%%%%%%%%%%%%%%%%%%%%%
% Jedes LaTeX-Dokument muss eine \documentclass-Deklaration enthalten0
%   Diese sorgt fuer das allgemeine Seiten-Layout, das Aussehen der 
%   Ueberschriften etc.
\documentclass[a4paper,oneside,DIV8,8pt]{scrartcl}
  
  %%%%%%%%%%%%%%%%%%%%%%%%%%%%%%%%%%%%%%%%%%%%%%%%%%%%%%%%%%%%%%%%%%%%%%%%%%
  % Einbinden weiterer Pakete
  \usepackage{german}    % fuer die deutschen Trennmuster
  % \usepackage{ngerman} % entsprechend fuer die neue Rechtschreibung
  \usepackage[latin1]{inputenc} % falls Sie Umlaute in den Quellen verwenden wollen
  \usepackage{amsmath}   % enthaelt nuetzliche Makros fuer Mathematik
  \usepackage{amsthm}    % fuer Saetze, Definitionen, Beweise, etc.
  \usepackage{relsize}   % fuer \smaller 
  \usepackage{german,longtable}
  \usepackage[T1]{fontenc}
  %\usepackage[utf8]{inputenc}
  \usepackage{listings}
  \usepackage{float}

  %%%%%%%%%%%%%%%%%%%%%%%%%%%%%%%%%%%%%%%%%%%%%%%%%%%%%%%%%%%%%%%%%%%%%%%%%%
  % Deklaration weiterer Makros
  \renewcommand{\labelitemi}{--}             % aendert die Symbole bei unnumerierten Aufzaehlungen
  \makeatletter                              % Fussnote ohne Symbol
    \def\blfootnote{\xdef\@thefnmark{}\@footnotetext}
  \newcommand{\handouttitle}[4]
   {\begin{center}
      \Large #4
    \end{center}

    %\bigskip

    \noindent
    #1, \textsf{#2}
    \hfill
    #3 
    \blfootnote{Testbericht, InfinityRun, {\smaller HFU}~Furtwangen}
  
    \noindent
    \rule{\linewidth}{.5pt}

    %\bigskip

    %\@afterindentfalse\@afterheading
   }
  %\makeatother
  \renewcommand{\sectfont}{\normalfont}       % aendert den Font fuer Ueberschriften

%%%%%%%%%%%%%%%%%%%%%%%%%%%%%%%%%%%%%%%%%%%%%%%%%%%%%%%%%%%%%%%%%%%%%%%%%%%%
% Anfang des eigentlichen Dokuments
\begin{document}
  % Titel fuer das Handout -- Sie koennen natuerlich auch selbst etwas entwerfen!
  \handouttitle{Name: Gruppe 4}
			   {Beruf: Studenten}
               {Datum: 21.12.2016}
               {Testbericht}
\section{Testumgebung}
\begin{table}[h]
	\centering
	\begin{tabular}{l|l}
		Browser & Chrome	\\ 
		Betriebssystem & Linux\\
		System & Laptop	\\ 
		Aufl�sung & Full-HD \\
	\end{tabular}
\end{table}
\section{Modultest}
\subsection{Steuerung}
\begin{table}[h]
	\centering
	\begin{tabular}{l|c|l}
		\textbf{}& \textbf{Erf�llt Ja/Nein?} & \textbf{Anmerkungen}\\
		Funktioniert & Ja &\\ 
		Intuitiv? & Ja & \\
	\end{tabular}
\end{table}
\subsection{Sound}
\begin{table}[h]
	\centering
	\begin{tabular}{l|c|l}
		\textbf{}& \textbf{Erf�llt Ja/Nein?} & \textbf{Anmerkungen}\\
		Funktioniert & Ja/Nein & Probleme bei dauerhafter Wiedergabe\\ 
		Passend? & Ja & \\
	\end{tabular}
\end{table}
\subsection{Spieloberfl�che}
\begin{table}[H]
	\centering
	\begin{tabular}{l|c|l}
		\textbf{}& \textbf{Erf�llt Ja/Nein?} & \textbf{Anmerkungen}\\
		Farbgestaltung passend? & Ja &\\ 
		Darstellung der Komponenten & Ja & \\
		\end{tabular}
\end{table}
\subsection{Spielverlauf}
\begin{table}[H]
	\centering
	\begin{tabular}{l|c|l}
		\textbf{}& \textbf{Erf�llt Ja/Nein?} & \textbf{Anmerkungen}\\
		Spielkonzept passend? & Ja &\\
		Funktioniert der Highscore & Ja &\\ 
		Schwierigkeit passend? & Ja & \\
	\end{tabular}
\end{table}
\section{Systemtest}
\textbf{Performance:}
Nicht durchgehend Fl�ssig Spielbar.
%%%%%%%%%%%%%%%%%%%%%%%%%%%%%%%%%%%%%%%%%%%%%%%%%%%%%%%%%%%%%%%%%%%%%%%%%%%%
% Ende des Dokuments -- alles, was nach dieser Zeile steht, wird 
% von LaTeX ignoriert!
\end{document}