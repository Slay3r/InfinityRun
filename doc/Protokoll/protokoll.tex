% Vorlage fuer Handouts
% zum Seminar "Diskrete Geometrie und Kombinatorik -- ein topologischer Zugang"
% im WS 2008/09
%
%%%%%%%%%%%%%%%%%%%%%%%%%%%%%%%%%%%%%%%%%%%%%%%%%%%%%%%%%%%%%%%%%%%%%%%%%%%%
% Allgemeine Hinweise
% - Halten Sie den LaTeX-Code so uebersichtlich wie moeglich;
%   (La)TeX-Fehlermeldungen sind oft kryptisch -- in einem ordentlich 
%   strukturierten Quellcode lassen sich Fehler leichter finden und 
%   beseitigen
%
%
%%%%%%%%%%%%%%%%%%%%%%%%%%%%%%%%%%%%%%%%%%%%%%%%%%%%%%%%%%%%%%%%%%%%%%%%%%%%
% Jedes LaTeX-Dokument muss eine \documentclass-Deklaration enthalten0
%   Diese sorgt fuer das allgemeine Seiten-Layout, das Aussehen der 
%   Ueberschriften etc.
\documentclass[a4paper,oneside,DIV8,10pt]{scrartcl}
  
  %%%%%%%%%%%%%%%%%%%%%%%%%%%%%%%%%%%%%%%%%%%%%%%%%%%%%%%%%%%%%%%%%%%%%%%%%%
  % Einbinden weiterer Pakete
  \usepackage{german}    % fuer die deutschen Trennmuster
  % \usepackage{ngerman} % entsprechend fuer die neue Rechtschreibung
  \usepackage[latin1]{inputenc} % falls Sie Umlaute in den Quellen verwenden wollen
  \usepackage{amsmath}   % enthaelt nuetzliche Makros fuer Mathematik
  \usepackage{amsthm}    % fuer Saetze, Definitionen, Beweise, etc.
  \usepackage{relsize}   % fuer \smaller 
  \usepackage{german,longtable}
  \usepackage[T1]{fontenc}
  %\usepackage[utf8]{inputenc}
  \usepackage{listings}
  \usepackage{float}

  %%%%%%%%%%%%%%%%%%%%%%%%%%%%%%%%%%%%%%%%%%%%%%%%%%%%%%%%%%%%%%%%%%%%%%%%%%
  % Deklaration weiterer Makros
  \renewcommand{\labelitemi}{--}             % aendert die Symbole bei unnumerierten Aufzaehlungen
  \makeatletter                              % Fussnote ohne Symbol
    \def\blfootnote{\xdef\@thefnmark{}\@footnotetext}
  % Titel des Handouts
  %   #1 Name des Vortragenden
  %   #2 email-Adresse 
  %   #3 Datum des Vortrags
  %   #4 Titel des Vortrags
  \newcommand{\handouttitle}[4]
   {\begin{center}
      \Large #4
    \end{center}

    \bigskip

    \noindent
    #1 (\textsf{#2})
    \hfill
    #3%
    \blfootnote{Protokoll, Informatik Workshop 
      WS~2016/2017, {\smaller HFU}~Furtwangen}
  
    \noindent
    \rule{\linewidth}{.5pt}

    \bigskip

    \@afterindentfalse\@afterheading
   }
  \makeatother
  \renewcommand{\sectfont}{\normalfont}       % aendert den Font fuer Ueberschriften

%%%%%%%%%%%%%%%%%%%%%%%%%%%%%%%%%%%%%%%%%%%%%%%%%%%%%%%%%%%%%%%%%%%%%%%%%%%%
% Anfang des eigentlichen Dokuments
\begin{document}
  % Titel fuer das Handout -- Sie koennen natuerlich auch selbst etwas entwerfen!
  \handouttitle{Gruppe 4}
				{Florian Durli, Johannes But, Jannik Ivosevic, Koray Emtekin, Marco Mayer}
               {21.~Dezember~2016, C 2.16, 07.45 - 11 Uhr}
               {Protokoll}
\section{Anwesenheit}
\begin{table}[h]
	\centering
	\begin{tabular}{l|c}
		\textbf{Name}& \textbf{Anwesenheit}\\
		Florian Durli & Ja	\\ 
		Johannes But & Ja	\\
		Koray Emtekin & Ja	\\ 
		Jannik Ivosevic & Ja \\
		Marco Mayer & Ja \\
	\end{tabular}
	\caption{Anwesenheit}
\end{table}
\section{Protokoll}
\subsection{Aufgabenverteilung}
\begin{table}[h]
	\centering
	\begin{tabular}{llc}
		\textbf{Name}& \textbf{Aufgabe}\\
		Florian Durli & Doku + Testplan	\\ 
		Johannes But & Vorlage Testbericht von dritten	\\
		Koray Emtekin & Vorlage Testbericht von dritten	\\ 
		Jannik Ivosevic & Doku + Testplan \\
		Marco Mayer & Vorlage Testbericht von dritten \\
	\end{tabular}
	\caption{Aufgabenverteilung}
\end{table}
%%%%%%%%%%%%%%%%%%%%%%%%%%%%%%%%%%%%%%%%%%%%%%%%%%%%%%%%%%%%%%%%%%%%%%%%%%%%
% Ende des Dokuments -- alles, was nach dieser Zeile steht, wird 
% von LaTeX ignoriert!
\end{document}